% \documentclass[a4paper,12pt]{extarticle}  % Supports 8pt, 9pt, 10pt, 11pt, 12pt, 14pt, 17pt and 20pt.
\documentclass[12pt]{article}
\setlength{\parindent}{0pt}
\setlength{\parskip}{0.1em} % Adjust '1em' to your preference


% \usepackage[a4paper]{geometry}  % Set the page size to be A4 as default
% \geometry{
%  left=15mm,   % Sets the left margin
%  right=15mm,  % Sets the right margin
%  top=15mm,    % Sets the top margin
%  bottom=15mm  % Sets the bottom margin
% }
\usepackage{geometry}
\geometry{
  left=15mm,   % Sets the left margin
  right=15mm,  % Sets the right margin
  top=15mm,    % Sets the top margin
  bottom=15mm  % Sets the bottom margin
}
\usepackage[utf8]{inputenc}
\usepackage[english]{babel}


\usepackage{graphicx} % Required for inserting images
%In order to put images in the document enter the following command:
%\includegraphics{path/to/image}
%put it in figure environment to add caption and label
%\begin{figure}[H]
%    \centering
%    \includegraphics{path/to/image}
%    \caption{Caption}
%    \label{fig:my_label}
%\end{figure}

%in order to put image alongside text enter the following command:
% \begin{minipage}{0.5\textwidth} % Adjust the width to fit your needs
%   \fbox{\includegraphics[width=\linewidth]{path/to/image}}
% \end{minipage}%
% \hfill % Optional: Adds horizontal space between the minipages
% \begin{minipage}{0.5\textwidth} % Adjust the width accordingly
%   Your text here
% \end{minipage}
%This sill put the image in a box and align it with the text, if you want to add a caption and label you can put the minipage in a figure environment. To remove the box remove the \fbox command.

\usepackage{pdfpages}
% \includepdf[pages=-]{path/to/pdf}


\usepackage{xcolor}

\usepackage{amsthm}
\usepackage{amsfonts} % or \usepackage{amssymb}
\usepackage{amsmath}
\usepackage{mathtools}

\usepackage{hyperref}
\usepackage{float}

% \usepackage{csquotes}  % Recommended for biblatex
% \usepackage[backend=biber, style=numeric]{biblatex}
% \addbibresource{references.bib}  % The filename of the bibliography
\usepackage{natbib}
\bibliographystyle{plainnat}


\usepackage{etoolbox}  % Ensure this package is included

% Define the \citeauthorplus command
\newrobustcmd{\citeauthorplus}[1]{\citeauthor{#1} [\citenum{#1}]}


% \theoremstyle{definition}
\newtheorem{definition}{Definition}
\newtheorem{theorem}{Theorem}
\newtheorem{lemma}{Lemma}
\newtheorem{remark}{Remark}
\newtheorem{example}{Example}
\newtheorem{reminder}{Reminder}
\newtheorem{assumption}{Assumption}
\newtheorem{note}{Note}
\newtheorem{proposition}{Proposition}


\usepackage[noend]{algpseudocode}
\usepackage{algorithm}

\algrenewcommand\algorithmicrequire{\textbf{Input:}}
\algrenewcommand\algorithmicensure{\textbf{Output:}}

\DeclareMathOperator*{\argmin}{arg\,min}
\DeclareMathOperator*{\argmax}{arg\,max}
\DeclarePairedDelimiter\abs{\lvert}{\rvert}%
\DeclarePairedDelimiter\norm{\lVert}{\rVert}%


\usepackage{xcolor}
\newcommand{\naomi}[1]{{\color{red}{Naomi: #1}}}
\newcommand{\gur}[1]{{\color{blue}{Gur: #1}}}
\newcommand{\todo}[1]{{\color{orange}{TODO: #1}}}



\title{Artificial Intelligence and Autonomous Systems 096208 - Abstract Submission}
\author{Gur Keinan 213635899, Naomi Derel 325324994}
\date{August 22, 2024}


\begin{document}

\maketitle

\section{Field of Research}

Our project aims to predict search progress utilizing Graph Neural Networks (GNNs). Following previous work, we define search progress as the fraction of the total search effort that has already been expended. More formally, given a search algorithm $A$ and a heuristic search problem $P$, the search progress of algorithm $A$ solving problem $P$ after expanding $Gen_A(P)$ nodes with $Rem_A(P; Gen_A(P))$ remaining nodes to be expanded is defined as:
\begin{equation}
  Prog_A(P) = \frac{Gen_A(P)}{Gen_A(P) + Rem_A(P; Gen_A(P))}
\end{equation}
Meaning - the fraction of the total search effort that has already been expended. \\

We hypothesis that GNNs can learn the underlying structure of the search space, and predict search progress more accurately than previous methods.

Our intuition is that the receptive field of a search node can be used to predict search progress, as the size and qualities of these nodes might help determine whether we are in some plateau of search, closer to the goal, etc. As previous work has shown promise in predicting search progress based on features with deep learning methods such as LSTMs, we are interested in exploring the potential of GNNs in this domain.

% \begin{enumerate}
%   \item Consider a search node with a large immediate receptive field. We would expect it to have a lower search progress, as the search algorithm opens up many new nodes close to the current node, meaning it needs more exploration to reach the goal.
%   \item Consider a search node with a small immediate receptive field. We would expect it to have a higher search progress, as the search algorithm is likely to be close to the goal \gur{maybe rephrase this} \naomi{Naomi this is your color, you can call it using \textbackslash naomi\{text\}}
% \end{enumerate}

\section{Literature Review}



\section{Work Domain}

Our project will be practical, focusing on creating aduquate datasets and training GNNs to predict search progress. We aim to create our own simple dataset of Blocksworld problems in STRIPS \cite{strips1971}, and use the A* search algorithm with different heuristics to solve them. We will then train a GNN to predict search progress based on the search nodes expanded by the A* algorithm.

\section{Work Outline}

\begin{algorithm}
  \caption{Sampling Blocksworld Instance}
  \label{alg:sample_blocksworld}
  \begin{algorithmic}[1]
    \Require Number of blocks $n$
    \Ensure Blocksworld instance
    \State Initialize $n$ blocks
    \State Sample two permutations $p_1 = (i_1, i_2, \ldots, i_n)$ and $p_2 = (j_1, j_2, \ldots, j_n)$ of $1, 2, \ldots, n$.
    \State Sample two binary vectors $b_1, b_2 \in \{0, 1\}^n$.

    \State In the initial configuration, place block $i_1$ on the table. For every $k = 2, 3, \ldots, n$, place block $i_k$ on block $i_{k-1}$ if $b_1[k] = 1$, otherwise place it on the table.
    \State In the goal configuration, place block $j_1$ on the table. For every $k = 2, 3, \ldots, n$, place block $j_k$ on block $j_{k-1}$ if $b_2[k] = 1$, otherwise place it on the table.

  \end{algorithmic}
\end{algorithm}




The following is a rough outline of our project:
\begin{enumerate}
  \item Create a dataset of Blocksworld problems. An instance of the Blocksworld problem consists of a set of blocks, initial and goal configurations of the blocks. The goal is to move the blocks from the initial configuration to the goal configuration using legal moves. We hope that by sampling initial and goal configurations randomly uniformly, we will get a diverse set of problems that will be challenging for the search algorithm. We refer the reader to Algorithm~\ref{alg:sample_blocksworld} for a method to efficiently sample Blocksworld instances.
  \item Implement the A* search algorithm with different heuristics to solve the Blocksworld problems. Possible options considered at this stage are the $h_{max}$, $h_{add}$, and $h_{n}$ heuristics, where $h_{n}$ is the number of blocks that are not in their correct position.
  \item Running the A* algorithm on the Blocksworld problems, we will collect the labeled data for our GNN - for each node we expand, we will record the search progress.
  \item Train a GNN to predict search progress based on the search nodes expanded by the A* algorithm. We will use the PyTorch Geometric library to implement the GNN.
\end{enumerate}

\section{Research Goals}

\paragraph{Reasonable Results} We expect that the GNN will be able to predict search progress more accurately than previous methods. We hope for improved or comparable results to the previous methods mentioned on \citet{sudry2022learning}. Maybe we will implement the previous methods and compare them to our GNN, but this is not guaranteed.

\paragraph{Interpretability} We hope that the GNN will be able to learn the underlying structure of the search space and provide insights into the search progress. We will analyze the GNN's predictions and try to understand what it has learned.

\paragraph{Generalization} We hope that the GNN will be able to generalize to new Blocksworld problems that it has not seen before, with a varying number of blocks. Within time constraints, we find it reasonable to limit the scope of our project to Blocksworld problems and A* search. 

\bibliography{references}


\end{document}


% Project Abstract Requirements
% Course: 096208 - AI and Autonomous Systems
% Due: 22nd August 2024
% Abstract Requirements
% 1. Group Members:
% • Name: [Group Member 1]
% • ID: [ID 1]
% • Name: [Group Member 2]
% • ID: [ID 2]
% 2. Field of Research: [Specify the specific field, e.g., Bidirectional Search, Memory Limited Search,
% Abstraction Heuristics, etc.]
% 3. Literature Review:
% • Provide a summary of at least 3 relevant papers or works related to your chosen field of
% research.
% • Paper 1: [Reference details and summary]
% • Paper 2: [Reference details and summary]
% • Paper 3: [Reference details and summary]
% 4. Work Domain:
% • Specify whether your work is theoretical, involves simulators, or both. If your work includes
% simulators, try to specify which simulators you intend to use and how.
% 5. Work Outline:
% • Provide a brief outline of your project’s structure and objectives.
% 6. Initial Ideas/Work/Results:
% • Summarize any initial ideas, work done so far, or preliminary results.
% 7. Additional Ideas (Not Guaranteed to Be Achieved):
% • List any additional ideas or goals that may be considered but are not guaranteed to be achieved
% within the project timeframe.
% 1
% ChatGPT
% Here’s the revised abstract template for your project, incorporating the requirements provided:


% Learning to Estimate Search Progress Using Sequence of States
% Matan Sudry, Erez Karpas
% Technion —Israel Institute of Technology, Haifa, Israel
% matansudry@campus.technion.ac.il, karpase@technion.ac.il
% Abstract
% Many problems of interest can be solved using heuristic
% search algorithms. When solving a heuristic search problem,
% we are often interested in estimating search progress, that is,
% how much longer until we have a solution. Previous work on
% search progress estimation derived formulas based on some
% relevant features that can be observed from the behavior of
% the search algorithm. In this paper, rather than manually deriving
% such formulas we leverage machine learning to automatically
% learn more accurate search progress predictors. We
% train a Long Short-Term Memory (LSTM) network, which
% takes as input sequences of nodes expanded by the search algorithm,
% and predicts how far along the search we are. Importantly,
% our approach still treats the search algorithm as a black
% box, and does not look into the contents of search nodes. An
% empirical evaluation shows our technique outperforms previous
% search progress estimation techniques.
% Introduction
% Many interesting and challenging real-world problems can
% be defined as search problems, where the objective is to find
% a path in a state space (which is often implicitly defined
% by a black-box successor generator) from some given initial
% node to a goal node. Often, we can use a heuristic function
% which estimates the distance from any given node to the
% goal to better guide this search. Many heuristic search algorithms
% have been developed to solve such problems, the
% most well-known of which are probably A (Hart, Nilsson,
% and Raphael 1968) and Greedy Best-First Search (GBFS)
% (Doran and Michie 1966).
% Search algorithms can have unpredictable behavior, as we
% typically do not know the topology of the search space (if we
% did, perhaps we would not need to use a search algorithm).
% While some theoretical results exist for cases where the
% search problem is described symbolically (Hoffmann 2011),
% these results do not apply for black-box search problems.
% Other work has addressed estimating search effort in terms
% of number of expanded nodes (Lelis, Stern, and Sturtevant
% 2014; Belov et al. 2017) or the cost of an optimal solution
% (Lelis et al. 2014) – we discuss these and others in more detail
% later. Importantly, these approaches are all offline – they
% give their predictions before the search starts.
% Copyright © 2022, Association for the Advancement of Artificial
% Intelligence (www.aaai.org). All rights reserved.
% We are interested in predicting remaining search effort online,
% that is, in estimating how much progress the search algorithm
% has already made. This problem was first introduced
% by Thayer, Stern, and Lelis (2012), who also derived some
% formulas for estimating search progress based on relatively
% simple search features, which we explain in more detail later.
% Search progress estimation is a useful tool not just for
% letting a user know whether they have time to make a cup
% of coffee before search completes, but also in making decisions
% about the search. For example, remaining search time
% estimates are used in situated temporal planning (Cashmore
% et al. 2018; Shperberg et al. 2021) to prioritize search nodes
% which have a higher probability to lead to a timely solution.
% Such estimates can also be used to decide when to perform a
% restart in a Branch-and-Bound search (Anderson et al. 2019)
% for solving a Mixed Integer Program.
% In this paper, we focus on search progress estimation (not
% its above mentioned applications), and leverage recent developments
% in machine learning to automatically learn more
% complex search predictors based on a given training set.
% Specifically, we view the search algorithm as a black-box
% that emits a sequence of nodes, and train a Long Short-
% Term Memory (LSTM) network (Hochreiter and Schmidhuber
% 1997) to predict remaining search effort. An empirical
% evaluation on a large set of planning domains shows that our
% technique achieves state-of-the-art performance in search
% progress estimation, outperforming previous work (Thayer,
% Stern, and Lelis 2012).
% Background
% In this section we review some of the necessary background
% on LSTM networks and heuristic search.
% Heuristic Search
% As previously mentioned, many problems of interest can be
% modeled as state-space search problems (Bonet and Geffner
% 2001). A state space is formally represented as a tuple T =
% hS; S0; SG; A; f; ci, where:
% • S is a finite and non-empty set of states,
% • S0 2 S is the initial state,
% • SG  S is a non-empty set of goal states,
% • A(s)  A denotes the actions applicable in each state
% s 2 S,
% Proceedings of the Thirty-Second International Conference on Automated Planning and Scheduling (ICAPS 2022)
% 362
% • f : S × A ! S is the state transition function, such that
% applying action a in state s leads to state f(s; a), and
% • c(s; a) is the cost of performing action a in state s
% Uninformed search algorithms such as Breadth-First
% Search (BFS) can be used to solve small state-space search
% problems. However, to solve larger problems, we typically
% also use a heuristic which estimates the distance from any
% given node to the goal. Formally, a heuristic is a function
% mapping the nodes S to some real number (or infinity to indicate
% a dead-end), that is h : S ! R+ [ {1}.
% Most search algorithms define some priority function
% f(s) over the nodes. Typically, f(s) is obtained from h(s),
% the heuristic estimate of the cost to go from s, and g(s), the
% cost of the best known path to s so far. For example, A
% (Hart, Nilsson, and Raphael 1968) expands the node with
% the lowest fA (s) = g(s) + h(s), while GBFS expands the
% node with the lowest fGBFS(s) = h(s).
% Search Progress Estimation
% We can now formally describe the problem we address in
% this paper – search progress estimation. Our definition here
% is adapted from previous work (Thayer, Stern, and Lelis
% 2012). Let A be a search algorithm, and P = hT; hi be a
% heuristic search problem, which consists of a search space T
% and a heuristic function h. Denote by EA(P) the total number
% of nodes expanded by search algorithm A while solving
% search problem P. Let RemA(P; GenA(P)) be the number
% of remaining nodes that are going to be expanded by
% A when solving P after A has already expanded GenA(P)
% nodes, that is, RemA(P; GenA(P)) = EA(P) − GenA(P).
% Definition 1 (Search Progress). The search progress of algorithm
% A solving problem P after expanding GenA(P)
% nodes is: ProgA(EA(P)) = GenA(P)
% GenA(P)+RemA(P;GenA(P))
% As Definition 1 shows, search progress is a number between
% 0 and 1, which indicates what fraction of the total
% search effort to solve the problem we have already expended.
% For example if we expanded node number 1,000
% out of 10,000 nodes that are going to be expanded in total,
% the search progress will be 1; 000=10; 000 = 0:1. This
% allows comparing search progress between problems of different
% sizes on the same scale.
% LSTM Networks
% Long Short-Term Memory (LSTM) networks (Hochreiter
% and Schmidhuber 1997) are a specific type of Recurrent
% Neural Network (Medsker and Jain 2001) – a deep learning
% approach which is especially suitable for learning from
% sequences. LSTMs were first introduced in order to solve
% the vanishing gradient problem in conventional Recurrent
% Neural Networks (Hochreiter and Schmidhuber 1997), and
% have been extensively used in natural language processing,
% e.g., (Sutskever, Vinyals, and Le 2014; Gers, Schmidhuber,
% and Cummins 2000).We now briefly review how LSTM networks
% are built, but refer the interested reader to a detailed
% survey (Yu et al. 2019) for more details.
% An LSTM network consists of several LSTM cells, arranged
% in sequence. Each LSTM cell has a hidden state, and
% 3 gates: input, forget and output, that manage the reading,
% writing and memory update, respectively. The cell learns
% gate weights and uses them to change the output and the
% hidden state based on the previous input in the sequence.
% The cell processes input xt together with the hidden state
% ht−1 (which is the result of processing the previous input)
% by first passing them to the forget gate, which has a Sigmoid
% activation function. The output of the forget gate is a number
% between 0 and 1 for each element in the cell state Ct – the
% information we are passing to the next LSTM cell.
% Next, the processing continues through the input gate,
% which has 2 activation functions, Sigmoid and tanh. The input
% is the same as the forget layer: xt and ht−1. The Sigmoid
% activation decides which values to update and the tanh activation
% creates the values themselves. The outputs from both
% of these are multiplied.
% Finally, processing continues to the output gate that gets
% xt and ht−1 as inputs and goes through a Sigmoid activation
% which decides what parts will be output. The output also
% goes through a tanh activation, which is multiplied by the
% output of the Sigmoid activation.
% This is all captured by the following equations, which describe
% the operation of each LSTM cell:
% • forget gate: ft = (Wf · [ht−1; xt] + bf )
% • input gate: it = (Wi · [ht−1; xt] + bi)
% • new cell state value: fCT = tanh(Wc · [ht−1; xt] + bC)
% • update cell state: CT = ft · Ct−1 + it · fCT
% • output gate: ot = (W0[ht−1; xt] + b0)
% • hidden state: ht = ot · tanh(Ct)
% Where W indicate learned weights and b a learned constant
% for different gates.
% Related Work
% Having described the necessary background, we can now
% review related work in this field. Some previous work attempted
% to predict search effort in specific domains, e.g.,
% predicting the number of nodes expanded by A for the 15-
% puzzle (Breyer and Korf 2008). Other work used sampling
% to estimate search effort (Lelis et al. 2014; Belov et al. 2017;
% Hutter et al. 2014) or the cost of a solution (Lelis, Stern,
% and Sturtevant 2014) in a more general setting. However, as
% previously mentioned, these are all offline methods, which
% attempt to make a prediction before the search starts. We
% now review in more detail the only prior work we are aware
% of that addresses online search progress estimation (Thayer,
% Stern, and Lelis 2012). Several search progress estimators
% were proposed, which we now describe, and later use as
% baselines in our empirical evaluation:
% Velocity-Based Search Speed Estimator (VeSP) The
% Velocity-Based Search Speed Estimator (VeSP) calculates
% the velocity of heuristic decrease during the search
% as:
% V =
% h0 − hmin
% GenA(P)
% where hmin is the minimum h-value of any nodes generated
% by the search so far and h0 is the heuristic value of
% 363
% Figure 1: Illustration of the nodes used as input for prediction.
% N is the node that was expanded last, N − 1 the node expanded before it, and so on. Parent() and Grandparent() are obtained
% by following the parent pointers for the indicated node.
% the initial state. It then uses the velocity V and the current
% hmin to estimate the remaining search effort (number of
% expanded nodes)
% SEV =
% hmin
% V
% Finally, it uses this estimate to predict search progress by
% using:
% V eSP(GenA(P)) =
% GenA(P)
% GenA(P) + SEV
% Vacillation-Based Search Speed Estimator (VaSP) The
% Vacillation-Based Search Speed Estimator (VaSP) differs
% from VeSP by using node serial numbers to estimate the
% expansion delay (Dionne, Thayer, and Ruml 2011), that
% is, the average number of expansions between when a
% node is generated and when it is expanded, denoted e.
% The expansion delay is combined with hmin to predict
% remaining search effort as:
% SEe = e · hmin
% As in VeSP, the estimate of search progress is:
% V aSP(GenA(P)) =
% GenA(P)
% GenA(P) + SEe
% Path-Based Progress Estimator (PBP) We first describe
% the Naive PBP (NPBP), which estimates search progress
% at a given node n by looking at the ratio between the cost
% until the current node (g(n)) and the total estimated cost
% (g(n) + h(n)),
% NPBP(Exp) =
% g(n)
% g(n) + h(n)
% The main problem with NPBP is that it depends on the
% current node, but in many cases there are other open
% nodes with higher progress. Thus the PBP estimator returns
% the maximum NPBP estimate among the nodes expanded
% so far.
% Distribution-Based Progress Estimator (DBP) The
% Distribution-Based Progress Estimator (DBP) estimates
% the search progress using data observed during search.
% It relies on counting how many nodes are expanded
% by the search for each value of d(n) – an estimate
% of remaining plan length. We remark that in unit-cost
% domains d(n) = h(n), and where action costs are
% non-uniform, d(n) can usually be estimated alongside
% h(n). Define c[di] as the count of how many nodes had
% value d(n) = di during search. DBP fits a second degree
% polynomial ^c to c, in order to estimate progress. The
% DBP estimate is then defined as:
% Prog(GenA(P)) =
% PGenA(P) m
% i=1 ^c[di]
% where m is the highest d-value with a count higher than
% 0.
% Learning to Predict Search Progress
% We now describe our technique for search progress estimation.
% Our technique exploits the fact that different heuristic
% searches can behave similarly to each other, especially
% when the search space, search algorithm, or heuristic are
% similar between the different searches. Thus, our technique
% uses supervised machine learning, training a search progress
% estimator on a set of already solved search problems. As
% our empirical evaluation (on a test set consisting of unseen
% search problems) will show, our technique is able to generalize
% across different problems, different domains, different
% search algorithms, and different heuristics.
% Our technique leverages the observation that any
% expansion-based forward search search algorithm can be
% seen as a black box which generates a sequence of nodes to
% be expanded, which we call a search history. Assume once
% search completes, the sequence of expanded nodes will be
% hs1; s2; : : : ; sni. After expandingmnodes, the partial search
% history is hs1; s2; : : : ; smi, which is the input to our predictor.
% The correct label that we would like to predict would
% then be m
% n .
% As we are attempting to predict from a sequence of nodes
% of varying length, it is natural to use a Recurrent Neural Network
% (Medsker and Jain 2001). In the remainder of this section
% we describe in more detail how we represent a partial
% search history as input features to the network, the network
% structure itself, and how we train it.
% Input Features: To represent a search history as input to
% our network, we look at the last k nodes that were expanded.
% However, for each of these nodes, it is also important to
% know something about where it came from – the path from
% the initial node to that node. Thus, for each of these k nodes,
% we also look at their ancestors m layers up. In this paper,
% we used used different values of k varying between 1 and
% 364
% 40 and fixed m = 2, resulting in collecting features from
% k · (m + 1) nodes, as illustrated in Figure 1.
% For each of these k · (m+1) nodes, we collect the following
% features:
% • g(n) – the cost of the best known path to node n.
% • h(n) – the heuristic estimate of the cost from n to the
% goal.
% • f(n) – the f-value for node n (which varies with the
% search algorithm).
% • b(n) – the branching factor of n, that is, the number of
% successors n has.
% • N(n) – the serial number of n, that is, how many nodes
% were expanded before n.We remark that this number was
% previously used to measure the expansion delay (Dionne,
% Thayer, and Ruml 2011).
% Additionally, we use some global features, which capture a
% snapshot of the current state of the search:
% • h0 = h(S0) – the heuristic value of the initial state. This
% value stays constant throughout the search, but is important
% for estimating the cost of the solution that will be
% returned.
% • hmin – the minimal h-value we have seen so far among
% the expanded nodes.
% • Nhmin – the number of nodes we expanded since the last
% time hmin was updated.
% • fmax – the maximum f value we have seen so far.
% These features are used in the previous work (Thayer,
% Stern, and Lelis 2012), with the exception of Nhmin, which
% is used to get some information about whether search seems
% to be stuck in a heuristic plateau – an area of the search space
% where the heuristic assigns similar values to all nodes, and
% thus heuristic guidance is not very useful. Search algorithms
% typically spend most of their time in heuristic plateaus, as
% otherwise, the heuristic would lead the search algorithm directly
% to the goal.
% We record the value of the global features at the time each
% node was expanded, and use these values for each of the
% last k states as features. These input features can be seen as
% a matrix of real numbers, of dimension 5k(m + 1) + 4k,
% which are used as the input (features) to our network. Note
% that for the first k nodes, we use feature values of 0 for the
% undefined nodes (i.e., nodes with a negative serial number).
% Network Structure: Our network1 consists of 15 LSTM
% layers, followed by a fully connected layer which reduces
% the dimension by half (from 15  k to 15  k=2), then a
% dropout layer with 50% dropout and Rectified Linear Units
% (ReLU) activations, and finally another fully connected layer
% which outputs our prediction – a single number. The parameters
% here (number of LSTM layers, dimensions, dropout
% rate, etc.) were determined by manual tuning over preliminary
% data. Automatically setting these parameters is future
% work.
% 1https://github.com/matansudry/Learning to Estimate Search
% Progress Using Sequence of States
% Domain VaSP VeSP RF OD
% Arprt(2) 26.4(10.7) 33.7(0.6) 54.9(1.3) 29.3(0.1)
% Blck(10) 29.8(7.6) 24.8(4.5) 39.8(5.7) 24.2(4.6)
% Depot(3) 31.2(4.5) 22.5(3.4) 35.7(8.5) 21.1(3.1)
% Elev(2) 33.1(2.7) 25.7(10.7) 36.3(4.7) 16.9(0.1)
% Frcll(3) 26.1(4.1) 28.0(5.7) 35.5(8.6) 20.5(5.1)
% OS(16) 32.8(5.7) 48.8(9.3) 34.7(16.5) 21.0(3.5)
% Pegsol(4) 37.3(11.7) 42.7(12.0) 22.7(11.1) 23.4(6.3)
% PSR(9) 38.9(9.0) 30.7(4.7) 24.9(8.6) 23.4(4.8)
% Rover(4) 33.9(2.7) 25.1(7.7) 29.3(8.5) 27.3(4.5)
% Sat(2) 27.5(4.5) 23.0(4.2) 23.9(10.4) 19.9(2.3)
% Scnlzr(1) 28.2(0.0) 37.0(0.0) 20.6(0.0) 24.0(0.0)
% Soko(19) 45.3(9.2) 31.7(7.3) 40.1(11.2) 29.4(5.0)
% TPP(7) 35.4(13.8) 38.4(8.1) 20.7(8.7) 19.9(4.1)
% Trns(6) 36.7(9.1) 36.0(9.6) 27.4(10.3) 21.9(4.5)
% WW(1) 44.0 31.5(0.0) 53.5(0.0) 18.5(0.0)
% Avgd(15) 33.4(5.9) 32.9(7.5) 33.2(9.6) 23.3(3.7)
% Avgp(89) 36.5(3.1) 35.3(3.3) 34.0(3.6) 24.0(5.5)
% GBFS with Lmcut
% Table 1: RMSE (in percent) and standard deviation
% in parentheses, using the OD regime and GBFS with
% Lmcut. Avgp is average over RMSE in each problem,
% and Avgd is the average over average RMSE in each
% domain. Results in bold are the best performance in
% this row. Arprt=Airport, Blck=Blocks, Elev=Elevators, Frcll=
% Freecell, OS=Openstacks, Parcp=Parcpriter, PSR=Psrsmall,
% Sat=Satellite, Scnlzr=Scanalyzer, Soko=Sokoban,
% Trns=Transport and WW=Woodworking
% Training: Given a planning problem, we first solve it using
% a planner running a given search algorithm and heuristic,
% and generate the true label (the correct search progress)
% for each node. During training, we sample 1,000 nodes uniformly
% at random from each problem in our dataset, in order
% to avoid bias to larger problems. This corresponds to the objective
% of optimizing the accuracy for a randomly chosen
% problem. If the objective had been to optimize the accuracy
% for a randomly chosen node, then it would be better to sample
% more nodes from the bigger problems. We use the node
% serial number to obtain the true label of search progress – for
% node n the true label is N(n)=N(gn), where gn is the goal
% node found by the search algorithm.
% We used Adam (Kingma and Ba 2014) as an optimiser
% with a learning rate of 0.001 and a batch size of 1024, and
% Mean Squared Error (MSE) as the loss function. Training is
% relatively fast, taking at most 12 minutes to train a searcheffort
% predictor on any given domain.
% Empirical Evaluation
% Having described our technique and the previous work on
% search progress estimation, we can now compare them empirically.
% We now describe this empirical evaluation.
% Benchmarks
% In order to compare search progress estimates on various
% problems with different heuristics and different search algorithms,
% we chose to use planning benchmarks from past
% International Planning Competitions, as this allows us to im-
% 365
% Domain VeSP DBP OD
% Arprt(6) 46.5(10.8) 44.8(8.5) 25.1(5.3)
% Blck(20) 38.0(6.6) 43.6(8.5) 21.8(3.3)
% Depot(19) 25.7(3.9) 27.2(6.8) 26.3(4.1)
% Elev(27) 27.9(6.9) 31.8(5.8) 27.3(3.1)
% Frcll(14) 18.0(7.1) 25.8(11.0) 20.2(4.0)
% Gripper(4) 57.7(0.0) 54.9(3.7) 24.3(3.9)
% Logistics(11) 17.9(2.9) 24.2(7.4) 26.7(3.9)
% Miconic(5) 42.6(5.1) 30.7(5.6) 20.6(2.9)
% OS(6) 29.4(2.5) 23.5(2.7) 23.8(3.0)
% Parcp(8) 24.7(3.3) 26.0(4.8) 23.6(5.9)
% Pegsol(18) 35.8(11.1) 48.8(6.6) 26.5(3.8)
% PSR(12) 48.3(14.1) 43.9(9.8) 25.3(4.2)
% Rover(4) 22.4(3.9) 35.8(7.6) 26.0(1.3)
% Sat(2) 33.3(0.0) 28.9(0.0) 24.7(0.2)
% Scnlzr(5) 37.2(14.4) 42.9(7.4) 26.1(3.7)
% Soko(20) 31.7(5.8) 40.7(7.8) 28.5(4.7)
% TPP(4) 25.8(4.3) 24.3(6.5) 23.9(3.4)
% Trns(8) 28.3(6.6) 29.3(7.7) 25.1(4.6)
% WW(8) 27.9(7.1) 45.0(12.9) 23.2(4.8)
% Zenotravel(3) 35.3(7.3) 27.1(1.7) 25.5(4.8)
% Avgd(20) 34.2(10.2) 36.2(9.7) 24.8(2.2)
% Avgp(204) 33.0(11.6) 37.3(11.7) 25.2(4.5)
% GBFS with hFF
% Table 2: RMSE (in percent) and standard deviation in parentheses,
% using the OD regime and GBFS with hFF . Results
% in bold are the best performance in this row. See caption of
% Table 1 for details
% plement our technique only once. Specifically, we extended
% Pyperplan (Alkhazraji et al. 2020) to output the features of
% the nodes it expands during search. This data is then used
% for training and testing of search progress estimators, as we
% describe next.
% Our dataset started with planning problems from 21 IPC
% domains – specifically, the benchmarks which are part of
% the Pyperplan repository. These domains have a total of 605
% planning problem instances. For each instance, we ran 4 configurations
% of Pyperplan, choosing one of two search algorithms
% (A* or GBFS) with one of two heuristics: hFF (Hoffmann
% and Nebel 2001) or Lmcut (Helmert and Domshlak
% 2009). We ran the planner with a time limit of 24 hours
% and a memory limit of 1,000,000 expanded nodes. From the
% solved instances we omitted the instances which were solved
% using less than 1,000 expanded nodes, as these are solved
% so quickly that search progress estimation is useless, and it
% would have made sampling 1,000 nodes from each instance
% difficult.
% Experimental Setup
% We compare our technique to the techniques introduced by
% previous work (Thayer, Stern, and Lelis 2012), which we
% described in detail above: VaSP, VeSP, PBP, and DBP. For
% VaSP, we used a moving average over the last 200 nodes
% to estimate expansion delay. For DBP we used NumPy to
% fit the polynomial to the data at hand. We also compared to
% flat machine learning approaches based on the same set of
% features as our LSTM. Specifically, we used Random Forest
% (RF) (Breiman 2001), which is a state-of-the-art classifier.
% All of our experiments were run on a server with 72 Intel
% Xeon E5-2695 CPUs (utilizing at most 18 processes in
% parallel (Tange et al. 2011). The deep learning was implemented
% in PyTorch, and run on 2 Nvidia Tesla M60 GPUs.
% We remark that this GPU was released more than 5 years
% ago and in current computers the GPU should be a more advanced
% version. On this hardware the inference takes 0.05ms
% per node and the collection time of the data is minor because
% the algorithms already calculate it.
% As our technique is based on learning from a training set,
% we examine three different training/test regimes:
% Other Domains (OD): In this regime, when we test on
% problems from some domain, we train on all problems
% from all other domains. This simulates the setting where
% we need to estimate search progress on a problem from a
% completely unknown domain.
% Same Domain (SD): In this regime, when we test on problems
% from some domain, we train only on problems from
% the same domain. To have some meaningful learning, we
% only used domains with more than 15 instances. For each
% domain, we split its instances into the even- and oddnumbered
% problems, to obtain two sets of roughly the
% same size, with roughly the same distribution on problem
% difficulty.We then trained a predictor on one set, and
% evaluated its performance on the other.
% Other Domains Tune Same (ODTS): In this regime, we
% first train a predictor based on instances from the other
% domains.We then take this trained model, and fine tune it
% on one half of the instances from the target domain (split,
% as before, to even and odd numbered problems). During
% this fine tuning we run the same training algorithm, except
% that the learning rate is 0.0001 and the batch size is
% 2048. As fine-tuning requires less data than SD, we used
% domains with more than 6 instances.
% We measure the accuracy of each search progress estimate
% prediction (for a given node) by the squared difference
% between the prediction and the true progress (which we
% have, since we use only solved problems). The accuracy of a
% search progress estimator over a given problem is the root of
% the average accuracy of its predictions over all the nodes expanded
% by the algorithm, that is the Root mean square error
% (RMSE). The accuracy for a set of problems (all problems
% in a domain, or all problems overall) is the average RMSE
% across these problems (giving equal weight to each problem,
% regardless of the number of nodes it required expanding).
% We also report the average of domain averages (that is, giving
% the same weight to each domain, regardless of how many
% problems it contains).
% Empirical Results
% We now describe the results of our empirical evaluation,
% starting with comparing the baselines to our approach using
% the OD regime. OD is the closest to these in terms of
% assumptions – it does not assume any knowledge about the
% problem it is used on. First, Figure 2 shows the accuracy
% using different values of K under the OD regime for different
% planner configurations. Unsurprisingly, as K increases
% so does accuracy, and so we present the results for K = 40
% 366
% Domain VaSP VeSP DBP RF OD
% Arprt(5) 31.3(12.3) 23.0(3.3) 21.1(3.4) 54.2(1.0) 29.2(0.2)
% Blck(13) 27.0(3.6) 31.8(8.6) 53.4(1.9) 29.1(11.3) 27.5(1.7)
% Depot(3) 28.9(2.3) 33.1(6.2) 45.0(3.6) 29.4(10.2) 24.4(0.7)
% Elev(11) 26.4(3.9) 29.1(5.6) 40.4(5.8) 25.4(5.4) 26.0(3.3)
% Frcll(1) 29.1(0.0) 23.7(0.0) 39.8(0.0) 24.9(0.0) 26.5(0.0)
% Gripper(4) 34.3(3.2) 21.7(1.4) 25.6(3.6) 35.1(15.9) 26.9(3.6)
% Logistics(5) 39.4(3.6) 28.7(2.8) 35.3(8.0) 41.9(6.9) 25.7(3.3)
% OS(5) 24.4(2.0) 40.3(3.6) 44.1(4.4) 32.6(8.4) 27.6(3.4)
% Parcp(6) 40.1(5.6) 42.2(2.8) 48.3(2.1) 36.6(13.3) 27.6(3.8)
% Pegsol(15) 34.7(2.7) 33.5(4.6) 48.4(2.3) 28.7(6.2) 27.5(3.9)
% PSR(15) 38.8(5.6) 31.3(3.2) 48.3(2.1) 27.2(10.2) 24.9(4.3)
% Rover(3) 28.6(1.8) 38.3(1.9) 46.7(1.6) 22.2(4.7) 26.8(1.7)
% Sat(2) 29.1(3.4) 27.1(4.4) 41.3(13.6) 42.3(18.9) 24.9(1.3)
% Scnlzr(3) 29.1(3.7) 29.8(3.8) 34.0(7.1) 22.2(12.2) 26.7(0.2)
% Soko(16) 35.1(8.3) 32.2(7.8) 45.7(7.8) 34.0(6.2) 27.2(2.7)
% TPP(1) 34.9(0.0) 47.7(0.0) 49.7(0.0) 20.4(0.0) 27.2(0.0)
% Trns(5) 26.5(1.8) 27.7(2.8) 40.4(5.9) 36.8(7.6) 25.6(3.4)
% WW(5) 54.6(3.5) 46.7(3.8) 51.6(7.6) 23.6(2.7) 27.3(3.9)
% Zenotravel(3) 28.7(4.6) 32.0(8.0) 36.0(8.3) 29.6(16.5) 24.9(14.1)
% Avgd(19) 33.4(7.1) 33.4(7.5) 42.7(8.6) 32.4(8.5) 26.6(1.3)
% Avgp(121) 34.0(8.2) 32.8(7.5) 44.8(9.2) 32.0(10.9) 26.7(3.2)
% A with Lmcut
% Table 3: RMSE (in percent) and standard deviation in parentheses, using the OD regime and A with Lmcut. Results in bold
% are the best performance in this row. See caption of Table 1 for details
% Domain VaSP VeSP PBP RF OD
% Arprt(8) 18.8(4.6) 16.1(6.4) 25.4(0.8) 34.3(14.3) 25.8(5.3)
% Blck(7) 36.2(0.6) 32.9(3.1) 28.7(0.1) 32.4(7.9) 26.3(1.5)
% Depot(1) 27.4(0.0) 19.9(0.0) 26.3(0.0) 22.4(0.0) 21.3(0.0)
% Elev(5) 31.8(1.4) 26.0(3.9) 27.9(0.4) 29.2(15.3) 28.0(0.8)
% Frcll(2) 34.7(4.2) 31.5(12.8) 27.6(0.8) 34.(7.6) 27.1(6.7)
% Gripper(4) 33.4(1.1) 55.2(2.8) 27.7(0.5) 22.4(8.3) 23.3(5.5)
% Logistics(9) 31.3(3.5) 28.1(2.2) 27.4(0.5) 22.3(7.4) 25.2(4.6)
% Miconic(4) 35.4(2.3) 24.3(2.3) 27.8(0.4) 23.8(8.5) 24.1(2.1)
% OS(5) 31.9(2.7) 18.5(1.0) 27.6(0.3) 20.8(6.7) 24.8(5.3)
% Parcp(5) 20.2(2.1) 24.1(2.1) 27.2(0.3) 30.3(4.7) 24.7(3.9)
% Pegsol(15) 42.1(0.7) 34.1(10.4) 28.4(0.3) 29.2(6.2) 25.5(2.6)
% PSR(16) 37.5(5.9) 49.2(14.0) 28.5(0.3) 26.6(11.2) 27.7(2.3)
% Sat(2) 30.3(3.8) 30.6(4.1) 28.2(0.5) 41.7(4.3) 26.1(3.2)
% Scnlzr(4) 35.9(1.9) 29.5(5.4) 28.4(0.6) 46.2(7.4) 28.2(1.4)
% Soko(16) 39.6(5.3) 30.4(6.2) 27.8(1.3) 37.7(9.4) 28.0(3.3)
% TPP(1) 31.6(0.0) 23.7(0.0) 26.9(0.0) 11.5(0.0) 23.6(0.0)
% Trns(6) 30.2(1.4) 25.9(7.0) 27.8(0.4) 28.9(4.10 26.1(4.4)
% WW(5) 36.3(0.6) 30.8(2.0) 28.5(0.2) 26.9(6.6) 27.6(2.1)
% Zenotravel(4) 33.2(3.3) 23.9(4.5) 27.8(0.4) 30.5(10.9) 27.3(1.5)
% Avgd(19) 33.0(5.8) 30.6(9.4) 27.7(0.8) 30.0(8.0) 25.9(1.9)
% Avgp(119) 34.8(7.3) 33.0(12.0) 27.8(1.0) 30.5(10.5) 26.4(3.5)
% A with hFF
% Table 4: RMSE (in percent) and standard deviation in parentheses, using the OD regime and A with hFF . Results in bold are
% the best performance in this row. See caption of Table 1 for details
% 367
% Figure 2: RMSE with Different K and Different Configurations,
% Best Baseline is the best RMSE from the other methods
% and from all the different configurations
% A/Lmcut A/hFF GBFS/Lmcut GBFS/hFF
% S 121 119 89 204
% B 9/19 5/18 11/15 15/20
% M 0.82 0.94 0.71 0.73
% Table 5: Experiment 1 summary on all configurations.
% S=Solved problems,B=Best accuracy, M=RMSE ours/best
% predictor.
% in the detailed tables in the remainder of this section. However,
% note that we outperform the previous techniques even
% with K = 20. Tables 1, 2, 3, and 4, compare our technique
% using the OD regime to the estimators proposed in previous
% work and to using the random forest classifier in all 4 configurations
% of heuristic and search algorithm. In each subtable,
% we display only the dominant algorithms, and omit
% those that did not get the best results in any domain. Note
% that some domains do not appear for some configuration –
% namely, those where no problems were solved.
% To summarize these results more concisely, Table 5 shows
% in how many domains OD was the best predictor, and the
% overall accuracy of our OD predictor vs. the best overall predictor
% from the previous work. Overall, OD was the most
% accurate, with better RMSE compared to the best predictor.
% These results hold whether we use the average over RMSE
% in each problem (Avgp), or the average over average RMSE
% in each domain (Avgd). Although OD got better results than
% the other methods, we want to understand where it works
% better and where it does not. Looking more closely at the
% results, we observe that larger problems had higher error.
% Figure 3 shows a scatter plot for each planner configuration.
% Each point corresponds to a solved problem, with the number
% of expanded nodes on the x-axis, and the RMSE of OD
% on the y-axis. We also add a linear regression line, showing
% a clear correlation. additionally we checked statistical
% significance using Kendall’s tau (Kendall 1938). The correlation
% we found is tau = 0:29; 0:23; 0:49; 0:36 for A with
% Figure 3: Correlation between number of expanded nodes
% and RMSE using OD in different configurations
% hFF , A with Lmcut, GBFS with hFF and GBFS with Lmcut
% respectively. In all cases, the p-value was less than 0:001,
% showing a statistically significant correlation between number
% of nodes and RMSE. We believe this is due to the fact
% that heuristics tends to be less accurate on the bigger (more
% difficult) problems, and since our search progress estimate
% relies on the heuristic, it is less accurate as well.
% Having seen that our technique, in the OD regime, outperforms
% previous work, we now turn to evaluating the impact
% of more specific training data by comparing the OD regime
% to the SD and ODTS regimes. Recall that in the SD regime
% our network is trained only on instances from the same domain,
% while in ODTS it is trained on instances from all other
% domains, and then fine-tuned on the domain being evaluated.
% Also recall that these regimes can only be used on domains
% which have enough data (15 solved instances for SD, 6 for
% ODTS), and thus the comparison is on fewer domains.
% Table 6 shows the results for this evaluation. In the table
% there are missing domains since in some configurations the
% number of problems the planner solved was below the required
% number we define for this experiment. Interestingly,
% ODTS outperforms SD in most cases, showing that the training
% data from other domains is important. ODTS also outperforms
% OD in most domains, as well as overall in each table,
% showing that domain-specific tuning does help. In some domains
% where OD outperforms ODTS, we observe that the
% split between the even and odd problems was not really balanced
% in terms of problem difficulty among the folds, and in
% some cases the number of expanded nodes in one fold was
% double that in the other. This could explain the poor behavior
% of ODTS.
% Generalization
% We conclude our empirical evaluation by evaluating how our
% predictor handles changing the search algorithm, the heuristic,
% or both. In this experiment, we trained our predictor (using
% the OD regime) on one choice of search algorithm and
% heuristic, and then tested its accuracy on another choice.
% 368
% Domain A
% OD SD ODTS
% Lmcut
% Blck 27.5(9.3) - 27.3(9.8)
% Elev 26.0(12.8) - 26.0(13.5)
% Parcp 27.6(13.7) - 27.3(15.2)
% Pegsol 27.5(14.0) 26.9(15.1) 27.0(21.4)
% PSR 24.9(14.7) 21.9(13.2) 21.3(12.3)
% Soko 27.2(12.0) 25.8(14.4) 25.1(13.4)
% Avgd(3) 26.6(8.1) 25.0(11.4) 25.8(10.6)
% Avgp(46) 26.7(12.8) 25.8(15.0) 25.5(16.3)
% hFF
% Arprt 25.8(5.3) - 22.8(5.6)
% Blck 26.3(1.5) - 24.2(2.6)
% Logistics 25.2(4.6) - 25.4(5.2)
% Pegsol 25.5(2.6) 24.9(2.7) 24.0(2.1)
% PSR 27.7(2.3) 23.1(5.6) 23.6(5.6)
% Soko 28.0(3.3) 27.5(3.3) 24.9(4.1)
% Trns 26.1(4.4) - 27.1(4.3)
% Avgd(3) 25.9(1.9) 25.2(9.3) 24.6(12.0)
% Avgp(47) 26.4(3.5) 27.5(4.5) 24.5(4.4)
% Domain GBFS
% OD SD ODTS
% Lmcut
% Blck 24.2(4.6) - 23.9(5.6)
% OS 21.0(3.5) 15.4(3.2) 13.4(2.3)
% PSR 23.4(4.8) - 23.0(6.5)
% Soko 29.4(5.0) 29.1(0.3) 29.1(2.3)
% TPP 19.9(4.1) - 24.0(9.6)
% Trns 21.9(4.5) - 25.0(6.1)
% Avgd(2) 25.2(3.7) 22.2(9.9) 21.3(5.2)
% Avgp(35) 24.0(5.5) 24.0(2.7) 23.8(2.7))
% hFF
% Arprt 25.1(5.3) - 22.7(5.0)
% Blck 21.8(3.3) 17.4(5.1) 17.5(5.6)
% Depot 26.3(4.1) - 24.7(6.3)
% Elev 27.3(3.1) 26.3(4.4) 26.5(5.0)
% Frcll 20.2(4.0) - 19.2(5.0)
% Logistics 26.7(3.9) - 24.8(3.1)
% OS 23.8(3.0) - 19.1(3.3)
% Parcp 23.6(5.9) - 21.7(6.8)
% Pegsol 26.5(3.8) 26.6(4.1) 26.8(4.9)
% PSR 25.3(4.2) - 26.8(4.3)
% Soko 28.5(4.7) 29.0(1.4) 29.6(4.6)
% Trns 25.1(4.6) - 25.7(6.8)
% WW 23.2(4.8) - 23.8(6.9)
% Avgd(4) 24.8(2.2) 25.2(5.3) 24.0(3.7)
% Avgp(85) 25.2(4.5) 25.3(6.0) 24.5(10.2)
% Table 6: RMSE (in percent) and standard deviation in parentheses
% with A/GBFS and Lmcut/hFF using all 3 regimes
% where Avgp is average RMSE over each problem and Avgd
% is the average over average RMSE in each domain. Results
% in bold are the best performance in this row. See caption of
% Table 1 for details
% With 2 search algorithms and 2 heuristics, we have 4 combinations,
% and thus 16 settings to evaluate.
% Table 7 shows the results of this cross domain evaluation.
% As expected, using the same configuration for training and
% in test leads to the best performance in almost all cases. The
% more interesting phenomenon is that when we change the algorithm
% and keep the heuristic the average RMSE increases
% by 1.43. When we change the heuristic and keep the algorithm,
% RMSE increases by 2.13, and changing both increases
% RMSE by 2.4. Note that, for any configuration, training on
% Test/Train G/L A/L G/H A/H
% G/L 24.0(5.4) 28.3(3.5) 28.3(4.4) 28.2(4.2)
% A/L 26.1(4.2) 26.7(3.2) 27.2(3.5) 27.6(3.3)
% G/H 25.7(5.8) 29.3(6.0) 25.2(4.5) 26.8(3.5)
% A/H 27.1(7.3) 29.2(5.5) 26.8(4.7) 26.4(3.5)
% Table 7: RMSE (in percent) and standard deviation in parentheses
% on generalization evaluation. results in bold are the
% best performance in this row. A=A, G=GBFS, H=hFF and
% L=Lmcut
% the polar opposite (different search algorithm and heuristic)
% still yields better accuracy than the best predictor from previous
% work. This shows that our predictor can generalize to
% a different search algorithm and heuristic without suffering
% too severe a decrease in performance.
% Discussion and Future Work
% In this paper we have described a novel approach to search
% progress estimation using deep learning. We have shown
% that our approach outperforms previous state-of-the-art approaches,
% and that it benefits from having access to better
% training data. It is also interesting to note that the previous
% search progress estimators (Thayer, Stern, and Lelis 2012),
% which were developed manually using extensive expertise in
% heuristic search and human creativity, are outperformed by
% a machine learning algorithm with very limited prior knowledge.
% Besides being an interesting problem, search progress estimation
% can be used to make decisions about search. For
% example, as previously mentioned, estimates of remaining
% search time have been used in situated temporal planning
% (Cashmore et al. 2018) and in Branch-and-Bound search
% (Anderson et al. 2019). Other applications are in anytime
% search (Dionne, Thayer, and Ruml 2011) or metareasoning
% (Shperberg et al. 2019, 2020, 2021). Specifically, the situated
% temporal planner (Shperberg et al. 2021) uses VaSP,
% which is outperformed by our technique. Thus, we expect
% replacing the search progress estimator there will improve
% the performance of the situated temporal planner.
% However, using search progress estimates to make decisions
% about search has a problem with self-reference – the
% predictions affect search decisions, which affect search time,
% which then affect the true search progress. We intend to address
% this challenge in future work by attempting to learn a
% series of predictors which converge to a fixed point which
% exhibits good performance. Finally, we remark that we employed
% fairly simple machine learning tools here. In future
% work we will examine whether we can improve performance
% even more by using different architectures of neural networks,
% or more informative features.
% Acknowledgments
% Supported by a grant from the Israeli Planning and Budgeting
% Committee.
% 369
% References
% Alkhazraji, Y.; Frorath, M.; Gr¨utzner, M.; Helmert, M.;
% Liebetraut, T.; Mattm¨uller, R.; Ortlieb, M.; Seipp, J.; Springenberg,
% T.; Stahl, P.; and W¨ulfing, J. 2020. Pyperplan.
% https://doi.org/10.5281/zenodo.3700819.
% Anderson, D.; Hendel, G.; Le Bodic, P.; and Viernickel,
% M. 2019. Clairvoyant restarts in branch-and-bound search
% using online tree-size estimation. In Proceedings of the
% Thirty-Third AAAI Conference on Artificial Intelligence and
% Thirty-First Innovative Applications of Artificial Intelligence
% Conference and Ninth AAAI Symposium on Educational Advances
% in Artificial Intelligence, 1427–1434.
% Belov, G.; Esler, S.; Fernando, D.; Le Bodic, P.; and
% Nemhauser, G. L. 2017. Estimating the size of search trees
% by sampling with domain knowledge. In IJCAI, 473–479.
% Bonet, B.; and Geffner, H. 2001. Planning as heuristic
% search. Artificial Intelligence, 129(1-2): 5–33.
% Breiman, L. 2001. Random forests. Machine learning,
% 45(1): 5–32.
% Breyer, T.; and Korf, R. 2008. Recent results in analyzing the
% performance of heuristic search. In Proceedings of the First
% International Workshop on Search in Artificial Intelligence
% and Robotics (held in conjunction with AAAI), 24–31.
% Cashmore, M.; Coles, A.; Cserna, B.; Karpas, E.; Magazzeni,
% D.; and Ruml, W. 2018. Temporal Planning while
% the Clock Ticks. In ICAPS, 39–46.
% Dionne, A. J.; Thayer, J. T.; and Ruml, W. 2011. Deadline-
% Aware Search Using On-Line Measures of Behavior. In
% SOCS, 39–46.
% Doran, J. E.; and Michie, D. 1966. Experiments with the
% graph traverser program. Proceedings of the Royal Society
% of London. Series A. Mathematical and Physical Sciences,
% 294(1437): 235–259.
% Gers, F. A.; Schmidhuber, J. A.; and Cummins, F. A. 2000.
% Learning to Forget: Continual Prediction with LSTM. Neural
% Computation, 12(10): 2451–2471.
% Hart, P. E.; Nilsson, N. J.; and Raphael, B. 1968. A Formal
% Basis for the Heuristic Determination of Minimum Cost
% Paths. IEEE Trans. Syst. Sci. Cybern., 4(2): 100–107.
% Helmert, M.; and Domshlak, C. 2009. Landmarks, critical
% paths and abstractions: what’s the difference anyway?
% In Proceedings of the Nineteenth International Conference
% on International Conference on Automated Planning and
% Scheduling, 162–169.
% Hochreiter, S.; and Schmidhuber, J. 1997. Long short-term
% memory. Neural computation, 9(8): 1735–1780.
% Hoffmann, J. 2011. Analyzing Search Topology Without
% Running Any Search: On the Connection Between Causal
% Graphs and h+. J. Artif. Intell. Res., 41: 155–229.
% Hoffmann, J.; and Nebel, B. 2001. The FF planning system:
% Fast plan generation through heuristic search. Journal of
% Artificial Intelligence Research, 14: 253–302.
% Hutter, F.; Xu, L.; Hoos, H. H.; and Leyton-Brown, K. 2014.
% Algorithm runtime prediction: Methods & evaluation. Artificial
% Intelligence, 206: 79–111.
% Kendall, M. G. 1938. A new measure of rank correlation.
% Biometrika, 30(1/2): 81–93.
% Kingma, D. P.; and Ba, J. 2014. Adam: A method for
% stochastic optimization. arXiv preprint arXiv:1412.6980.
% Lelis, L. H.; Stern, R.; Felner, A.; Zilles, S.; and Holte,
% R. C. 2014. Predicting optimal solution cost with conditional
% probabilities. Annals of Mathematics and Artificial
% Intelligence, 72(3): 267–295.
% Lelis, L. H.; Stern, R.; and Sturtevant, N. R. 2014. Estimating
% search tree size with duplicate detection. In Seventh
% Annual Symposium on Combinatorial Search.
% Medsker, L. R.; and Jain, L. 2001. Recurrent neural networks.
% Design and Applications, 5.
% Shperberg, S. S.; Coles, A.; Cserna, B.; Karpas, E.; Ruml,
% W.; and Shimony, S. E. 2019. Allocating Planning Effort
% When Actions Expire. In AAAI, 2371–2378. AAAI Press.
% Shperberg, S. S.; Coles, A.; Karpas, E.; Ruml, W.; and
% Shimony, S. E. 2021. Situated Temporal Planning Using
% Deadline-aware Metareasoning. In ICAPS, 340–348. AAAI
% Press.
% Shperberg, S. S.; Coles, A.; Karpas, E.; Shimony, S. E.; and
% Ruml,W. 2020. Trading Plan Cost for Timeliness in Situated
% Temporal Planning. In Bessiere, C., ed., IJCAI, 4176–4182.
% ijcai.org.
% Sutskever, I.; Vinyals, O.; and Le, Q. V. 2014. Sequence
% to sequence learning with neural networks. In Proceedings
% of the 27th International Conference on Neural Information
% Processing Systems-Volume 2, 3104–3112.
% Tange, O.; et al. 2011. Gnu parallel-the command-line
% power tool. The USENIX Magazine, 36(1): 42–47.
% Thayer, J. T.; Stern, R.; and Lelis, L. H. 2012. AreWe There
% Yet?-Estimating Search Progress. In SOCS.
% Yu, Y.; Si, X.; Hu, C.; and Zhang, J. 2019. A review of recurrent
% neural networks: LSTM cells and network architectures.
% Neural computation, 31(7): 1235–1270.
% 370